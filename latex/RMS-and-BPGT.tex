\documentclass[11pt]{article}

\usepackage[numbers,sort&compress]{natbib}  
%% Daniel added this, should help citations look nicer. You may need to delete temp files and rebuild the latex document from a clean start.

\newcommand{\daniel}[1]{{\textbf{{\small{\color{magenta}DL}: #1{\color{magenta}$\circ$}}}}} 
\newcommand{\owen}[1]{\textbf{{\small{\color{red}OK}: #1{\color{red}$\circ$}}}} 
\newcommand{\ed}[1]{\textbf{{\small{\color{blue}ED}: #1{\color{blue}$\circ$}}}}

%\renewcommand{\daniel}[1]{}
%\renewcommand{\owen}[1]{}
%\renewcommand{\ed}[1]{}


% Use utf-8 encoding for foreign characters
\usepackage[utf8]{inputenc}
\usepackage{url}
% Setup for fullpage use
\usepackage{fullpage}
\usepackage{color}
\usepackage{subfig}
\usepackage{hyperref}
\usepackage{algorithm}
\usepackage{algorithmic}

% Uncomment some of the following if you use the features
%
% Running Headers and footers
%\usepackage{fancyhdr}

% Multipart figures
%\usepackage{subfigure}

% More symbols
%\usepackage{amsmath}
%\usepackage{amssymb}
%\usepackage{latexsym}

% Surround parts of graphics with box
\usepackage{boxedminipage}

% Package for including code in the document
\usepackage{listings}

% If you want to generate a toc for each chapter (use with book)
\usepackage{minitoc}

% This is now the recommended way for checking for PDFLaTeX:
\usepackage{ifpdf}

%\newif\ifpdf
%\ifx\pdfoutput\undefined
%\pdffalse % we are not running PDFLaTeX
%\else
%\pdfoutput=1 % we are running PDFLaTeX
%\pdftrue
%\fi

% \ifpdf
\usepackage[pdftex]{graphicx}
% \else
% \usepackage{graphicx}
% \fi

\title{RMS and Backpropagation for Feedforward Neural Networks}
\author{Eduardo Gutarra}

% \date{2010--06--13}

\begin{document}
	
\ifpdf
\DeclareGraphicsExtensions{.pdf, .jpg, .tif}
\else
\DeclareGraphicsExtensions{.eps, .jpg}
\fi
	
\maketitle
	
\section{Introduction} % (fold)
\label{sec:introduction}

Artificial neural networks are computational models that mimic the architecture, structure and/or functional aspects of biological
neural networks such as the human brain. They are comprised of multiple processing elements called neurons which are interconnected
through links. Often, these links have a weights associated to them called synaptic weights. These weights scale the signals received
for different neurons, allowing the network to process patterns to generate an output pattern. Neurons are often grouped together in
layers or slabs and a neural network may be composed of 1 or more of these~\cite{skapura}.

These neurons work in parallel to perform various complex tasks. Each Neuron aggregates input that is scales the synaptic weights and
calculates an output with an activation function. The inputs to the neuron may be outputs from other neurons or the environment. Neural
networks possess important advantages and capabilities as computational models those include their capability to solve problems of
nonlinear nature, input-output mapping and adaptivity~\cite{Haykin:1994:NNC:541500}. They have also been applied successfully in the
following categories of applications:

\begin{itemize}
\item Function approximation or regression analysis
\item Classification, including pattern and sequence recognition, novelty detection and sequential decision making.
\item Data processing, including filtering, clustering, blind signal separation and compression.
\end{itemize}

The processing of a neural network begins when an external pattern made up of signals is captured by an input layer.

One of the greatest advantages of using a neural network is that we do not program it with a configuration to solve a problem. It is
actually programmed to learn and adapt to solve a problem. Neural networks have two types of learning supervised and unsupervised
learning. Supervised learning consists in giving a network a set of points and allowing to correct its prediction by giving it the
expected output. Unsupervised learning is often used when the network cannot be provided with a target output.

Our interest in this report is confined to a class of neural networks which emulate the process of learning. In this report we will
examine two different algorithms for training feedforward neural networks. One algorithm is the Backpropagation algorithm and the other
is the Root Mean Square (RMS) Minimization algorithm.

\section{Feedforward Neural Networks} % (fold)
\label{sec:feedforward_neural_networks}

In a feedforward neural network only neurons of adjacent layers are interconnected and the orientation is always forward. Neurons are
only interconnected forwards. Every layer of the neural network has connections to the next layer and there are no connections oriented
backwards.

The feed forward neural network begins with an input layer which may be connected to a hidden layer or directly to an output layer. The
first layer is the input layer. It directly connects to the external environment through which the input patterns are presented to the
network. The last layer is the output layer which produces the output pattern to the external environment. All other layers are
considered hidden layers, and may or may not be present. The input layer provides the means through which the external environment can
affect it.



% section feedforward_neural_networks (end)

% section introduction (end)


\section{Algorithms and Procedures} % (fold)
\label{sec:algorithms_and_procedures}

\subsection{Data Structure} % (fold)
\label{sub:data_structure}

% subsection data_structure (end)

\subsection{Algorithms} % (fold)
\label{sub:algorithms}

\subsection{Backpropagation training algorithm} % (fold)
\label{sub:backpropagation_training_algorithm}

Backpropagation is a method of supervised learning. It is used to train our feedforward neural network. To use the backpropagation
algorithm we provide it with both example inputs and target outputs. The outputs generated by the neural network are then compared
against the target outputs of the given example. Using the target outputs, the backpropagation training algorithm then calculates error
and adjusts the weights of the various layers backwards from the output layer to the input layer.

Backpropagation is often used to train feedforward neural networks; however it can be used to train other types of networks, and
likewise feedforward networks may be trained with other methods. In this report, we only examine the case where the backpropagation
training algorithm is applied to a feedforward neural network.

$E$ is a set of examples
$e$ is a single example
$I(e)$ set of inputs in a single example
$T(e)$ set of target outputs for a given example
$O(e)$ are the target outputs of the example
Each iteration of this loop we will call an epoch

% \begin{algorithmic}
% 	\FOR {$e \in E$} 
% 		\STATE $O(e) \gets$ FeedForward($I(e)$)
% 		\STATE CalculateOutputDeltas($O(e)$, $T(e)$)
% 		\STATE CalculateInternalDeltas
% 		\STATE UpdateWeights
% 	\ENDFOR
% \end{algorithmic}

% \begin{algorithm}
% {
% \begin{algorithmic}[1]
% \STATE \textbf{input:}  Unsorted table $t$ with $n$ rows and $c$ columns.
% \STATE \textbf{output:} a sorted table
% \STATE Form
% $K$~different versions of $t$, sorted differently: $t^{(1)},t^{(2)},\dots, t^{(K)}$ 
% 
% \STATE $\beta \leftarrow $ empty list
% \STATE pick an element in $t^{(1)}$ randomly, add it to $\beta$ and remove it from all $t^{(i)}$'s
% \WHILE{$\mathrm{size}(\beta)<n$ }
% \STATE let $r$ be the latest element added to $\beta$
% \STATE Given $i\in \{1,2,\dots,K\}$, there are up to two neighbors in sorted order within list $t^{(i)}$; out of up to $2K$ such neighbors, pick a nearest neighbor $r'$ to $r$ in Hamming distance.
% \STATE Add $r'$ to $\beta$ and remove it from all $t^{(i)}$'s
% \ENDWHILE
% \STATE \textbf{return} $\beta$
% \end{algorithmic}
% }
% \caption{\label{algo:multiplelists}The \textsc{Multiple Lists} heuristic}

\begin{algorithm}
{
	\textbf{Main algorithm:}
\begin{algorithmic}[1]
% \STATE \textbf{input:} a table $t$ and a block size $p$
% \STATE \textbf{output:} a sorted table
% %\STATE $\alpha \leftarrow $the columns in non-decreasing order of cardinality
% %\STATE $\beta \leftarrow $ empty list \owen{beta is unused}\daniel{leftover from a copy/paste?}
% \STATE each value in each tuple is assigned a Boolean value (initially false) indicating whether the value is DC
% \FOR{$k=1,2,\dots, c-1$}
% %\STATE append column $k$ to list $\beta$
% \STATE sort table $t$ %on the first $k$~columns %\owen{$\beta$, else $\beta$ unused} %the first $k$~columns 
% using the DC order with parameter $k$ (see below)
% \FOR{every run of identical values in column $k$}
% \STATE Mark %$x-\lfloor x/p \rfloor p$~values 
%                                        $x \bmod p$~values 
% as DC where $x$ is the length of the run 
% \ENDFOR
% \ENDFOR
% \STATE sort $t$ using the DC order with $k=c$
% \STATE \textbf{return} $t$
	\FOR {$e \in E$} 
		\STATE $O(e) \gets$ FeedForward($I(e)$)
		\STATE CalculateOutputDeltas($O(e)$, $T(e)$)
		\STATE CalculateInternalDeltas
		\STATE UpdateWeights
	\ENDFOR
\end{algorithmic}
}
%\hrule
\textbf{DC order:}
{
\begin{algorithmic}[1]
\STATE \textbf{input:} two $c$-tuples $x,y$, each component of each tuple might be marked as DC; a positive integer $k$ (we assume $k-1\leq c$)
\STATE \textbf{output:} 1,0,-1 depending on whether $x$ is greater, equal or smaller than $y$ 
\FOR{$i = 1,2,\dots,k-1$}
\IF{neither $x_i$ nor $y_i$ are DC}
\STATE \textbf{return} 1 if $x_i>y_i$ or -1 if $x_i<y_i$
\ELSIF{$y_i$ is DC}
\STATE \textbf{return} 1
\ELSIF{$x_i$ is DC}
\STATE \textbf{return} -1
\ENDIF
\ENDFOR
\FOR{$i = k,k+1,\dots,c$}
\STATE \textbf{return} 1 if $x_i>y_i$ , -1 if $x_i<y_i$
\ENDFOR
\STATE \textbf{return} 0 
\end{algorithmic}
}
\caption{\label{algo:dcsort}The Backpropagation Training Algorithm %\daniel{I tried to clean up further the pseudocode.}%\daniel{Pseudocode was buggy as remarked by Owen. }
%\owen{wonder whether we should revisit our for loop pseudocode syntax, which
%suggests that $i$ could take values from the set in any order.}\daniel{Fixed.}
}

\end{algorithm}


% subsection backpropagation_training_algorithm (end)



%\owen{wonder whether we should revisit our for loop pseudocode syntax, which
%suggests that $i$ could take values from the set in any order.}\daniel{Fixed.}


% subsection algorithms (end)

\section{Procedure} % (fold)
\label{sec:procedure}

% section procedure (end)

% section algorithms_and_procedures (end)

\section{Experiments and Results} % (fold)
\label{sec:results}

\subsection{Time Complexity} % (fold)
\label{sub:time_complexity}

To determine the number of operations performed in each epoch of the learning algorithm we count the weights between the layers of the
neural network. To do this we consider that the number of weights between two layers of neurons is the product of the number of neurons
of each layer. Therefore the total number of weights for the entire network is $\sum_{i=0}^{n-1}N_{i}N_{i+1}$ where $N_{i}$ is the
number of neurons in each layer, and $n$ is the total number of layers. For the backpropagation algorithm we obtained that the number
of operations in a single epoch depends on the number of weights and bias. Therefore it has linear complexity $O(n)$ with respect to
the number of neurons. In the RMS algorithm the number of operations in a single epoch of the RMS algorithm is squared because we run
the feedforward algorithm for each change we do on a single weight.

% subsection time_complexity (end)

% section results (end)

\section{Conclusion} % (fold)
\label{sec:conclusion}

The Backpropagation algorithm has time complexity within $O(n)$, which is faster than the RMS Minimization algorithm with a time
complexity of $O(n^{2})$ when varying the number of neurons in the middle layer. 

The Backpropagation algorithm reached a smaller error for the simple examples that the RMS Minimization algorithm.

As observed in some results, the neural network may get stuck in local minima unable to arrive to better solutions. We noticed this in
both the Backpropagation and RMS Minimization algorithms.

% section conclusion (end)	
    
\bibliographystyle{plain}
\bibliography{../bib/eds}
\end{document} 